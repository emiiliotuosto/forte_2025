\renewcommand{\partyP}[1][p]{\mathsf{\textcolor{BrickRed}{#1}}}
\renewcommand{\roleR}[1][R]{\mathsf{\textcolor{BrickRed}{#1}}}
\renewcommand{\roleO}{\roleR[O]}
\renewcommand{\roleB}{\roleR[B]}
\renewcommand{\partyB}{\partyP[b]}
\renewcommand{\partyO}{\partyP[o]}
\renewcommand{\contractC}[1][]{\mathsf{\textcolor{CadetBlue}{c}}_{#1}}
\renewcommand{\var}[1]{\mathsf{\textcolor{NavyBlue}{#1}}}
\renewcommand{\exprE}[1][]{\mathsf{\textcolor{orange}{e}}_{#1}}
%
\begin{frame}[<+->]{Basic concepts and notation}
  \emph{Participants} $\partyP,\partyP',\hdots$
  \vpause
  have \emph{roles} $\roleR,\roleR',\hdots$
  \vpause
  cooperate through a \emph{coordinator} $\contractC$ which is
  \vpause
  basically an object with fields and \quo{methods}:
  \vpause
  \begin{itemize}
  \item $\varCX,\varCY,\hdots$ represent sorted \emph{state variables}
	 of $\contractC$ (sort include 'participant' and usual data types
	 such as 'int', 'bool', etc.)
  \item $\functionCF,\functionCFi,\hdots$ which are the functions
	 operation admitted by $\contractC$
  \end{itemize}
  \vpause \emph{Assignment} $\varCX := \exprE$ where $\exprE$ is a
  standard syntax of \emph{pure} expressions\footnote{Plus the $\old$
	 qualifier for state variables as
	 in~\cite{DBLP:books/ph/Meyer90,DBLP:books/ph/Meyer91}}; let
  $\Assign, \Assigni, \hdots$ range over finite sets of assignments
  where each variable can be assigned at most once
  %
  \note{%
	 In every assignment~$\varCX := \exprE$ data variables occurring in
	 $\exprE$ must have the $\old$ qualifier to refer to their value
	 before the assignments.
	 \\[1em]
	 We adapt the mechanism based on the $\old$ keyword from the Eiffel
	 language~\cite{DBLP:books/ph/Meyer91} which, as explained
	 in~\cite{DBLP:books/ph/Meyer90} is necessary to render assignments
	 into logical formulae since e.g.,
	 $x=x+1 \Leftrightarrow \mathsf{False}$.
	 %
	 This will be used in \cref{def:consistency}.
	 %
  }
\end{frame}

\begin{frame}[t]{Data-Aware FSMs}
  \modelnames are finite-state machines whose transitions are decorated with specific labels
  \\[1em]
  Here are possible transitions of \modelnames\footnote{See~\cite[Def. 1]{akmrt24} for the formal definition}
  \\[1em]
  \begin{tikzpicture}[dafsm]
		\node (dummy) {};
		\node<2->[state, right = 7cm of dummy] (q0) {};
		\node<2->[legend, right = .3cm of q0] (c0) {
		  initial state of the contract $\contractC$ freshly created by $\partyP$
		  with state variables $\varCX[i]$ initialised by the assignments $\varCX[i] := \exprE[i]$
		};
		\note<2>{each state variable is declared and initialises with type-consistent expressions
		  \\[1em]
		  $\init$ is a \quo{build-in} function name
		}
		\node<3->[state] (q1) [below = 1.5cm of dummy] {};
		\node<3->[state] (q1') [right = 7cm of q1] {};
		\node<3->[legend, below = of c0, yshift=.8cm] (c1) {
		  where $\guardG$ (ie a boolean expression) is a guard and \\
		  \mytab[xxxxxx]$\modparty \bnfdef \new \bnfmid \any \bnfmid \partyP$\\
		  is a qualified participant invoking operation $\functionF$ with parameters $\varCY[i]$\\
		  state variables are reassigned according to $\Assign$ if the invocation is successful
		};
		\note<3>{\ \\[1em]
		  % 
		  $\guardG$ predicates over state variables and formal
		  parameters; guards have to be satisfied in order for the
		  invocation to be enabled: an invocation that makes the guard
		  false is \emph{rejected}
		  %
		  \\[1em]
		  $\new$ specifies that $\partyP$ must be a fresh participant with role $\roleR$
		  \\
		  $\any$ qualifies $\partyP$ as an existing participant with role $\roleR$
		  \\
		  $\partyP$ we refer to a participant in the scope of a binder
		  \\
		  invocations from non-suitable callers are \emph{rejected}
		  \\[1em]
		  %
		  the variables occurring in the right-hand side of assignments
		  in $\Assign$ are either state variables or parameters of the
		  invocation
		  %
		}
		\node<4->[state] (q2) [below = 1.5cm of q1] {};
		\node<4->[state,accepting] (q2') [right = 7cm of q2] {};
		\node<4->[legend, below = of c1, yshift=.6cm] (c2) {
		  accepting states denoted as usual
		};
		\path<2-> (dummy) edge[left] node[above, align=center] {
		  \dafsmlabel[][{\new[\partyP][\roleR]}][\init][{\contractC, \cdots \varCX[i]:T_i \cdots}][{\{\cdots \varCX[i] := \exprE[i] \cdots\}}]
		} (q0);
		%
		\path<3-> (q1) edge[right] node[sloped, anchor=center, above] {\dafsmlabel} (q1');
		%
		\path<4-> (q2) edge[right] node[sloped, anchor=center, above] {$\alabel$} (q2');
	 \end{tikzpicture}
\end{frame}


\begin{frame}
  \frametitle{Exercise: modelling}
  Give a \modelname for the following contract protocol:

  \note{let them play with qualified participants}
\end{frame}

\begin{frame}[t]{Not all \modelnames \quo{make sense}}
  \myitem{\raisebox{6pt}{
		\tikz[dafsm,node distance = 4cm]{
		  \node[state] (S0)      {};
		  \node (S_) [left of=S0] {};
		  \node[state,accepting] (S1) [right of=S0] {};
		  % 
		  \path
		  (S_) edge node {\dafsmlabel[][{\new[\partyO][\roleO]}][\init][\contractC][]} (S0)
		  (S0) edge node {\dafsmlabel[][\partyP][@][][]}
		  (S1);
		}
	 }
  }{free names}[\ ]
  \myitem{\raisebox{6pt}{
		\tikz[dafsm, node distance = 4cm]{
		  \node[state] (S0)      {};
		  \node (S_) [left of=S0] {};
		  \node[state,accepting] (S1) [right of=S0] {};
		  % 
		  \path
		  (S_) edge node {\dafsmlabel[][{\new[\partyO][\roleO]}][\init][\contractC][]} (S0)
		  (S0) edge node {\dafsmlabel[][\any][@][][]} (S1);
		}
	 }
  }{role emptyness}[\ ]
  \myitem{\raisebox{6pt}{
		\tikz[dafsm, node distance = 4cm]{
		  \node[state] (S0)      {};
		  \node (S_) [left of=S0] {};
		  \node[state,accepting] (S1) [right of=S0] {};
		  % 
		  \path
		  (S_) edge node {\dafsmlabel[][{\new[\partyO][\roleO]}][\init][\contractC][\varCX := 0]} (S0)
		  (S0) edge node {\dafsmlabel[\varCX > 0][\any][@][][]} (S1);
		}
	 }
  }{no progress}[\ ]

  \vpause

  Save name freeness, the other properties are undecidable in general,
  so we'll look for sufficient conditions on \modelnames ensuring role
  non-emptyness and progress.
\end{frame}

\begin{frame}{Closed \modelnames}
  \emph{Binders:} parameter declarations in function call, $\new$, and $\any$
  \vpause
  $\partyP$ is \emph{bound} in \dafsmtx if, for some role $\roleR$,
  \[
	 \modparty = \new[\partyP][\roleR] \qand[or]
	 \modparty = \any \qand[or]
	 \text{there is } i \text{ s.t. } \varY[i] = \partyP \text{ and } T_i = \roleR
  \]
  \vpause
  The occurrence of $\partyP$ is \emph{bound} in a path
  $\pathS \ \dafsmtx[@][@][\partyP] \ \pathS'$ if $\partyP$ is bound in a
  transition of $\pathS$
  \vpause
  A \modelname is \emph{closed} if all occurrences of participant
  variables are bound in the paths of the \modelname they occur on
\end{frame}

\begin{frame}{Roles non-emptyness}
  A transition \dafsmtx[@][@][\modparty] \emph{expands} role $\roleR$ if $\modparty = \new$
  or there is $i$ s.t. $\varY[i] = \partyP$ and $T_i = \roleR$
  \vfill
  Role $\roleR$ is \emph{expanded} in a path $\pathS \ \dafsmtx[@][@][\any] \ \pathS'$ if
  a transition in $\pathS$ expands $\roleR$
  \vfill
  A \modelname \emph{expands} $\roleR$ if all its paths expand
  $\roleR$ and is \emph{(strongly) empty-role free} if it expands all
  its roles
\end{frame}

\begin{frame}{Progress}
  A \modelname with state variables
  $X = \{\varCX[1], \ldots, \varCX[n]\}$ is \emph{consistent} if it is closed and the
  following implication holds for each transition
  \dafsmtx[@][@][@][@][@][@][\stateS]
  \[
  	 \forall (\varCX, \old\ \varCX)_{\varCX \in X} \
	 \exists (y)_{y \in Y}\qst
	 (\guardG\subst{\old\ \varCX}{\varCX}_{\varCX \in X} \ \land\ \guardG[\Assign]) \implies \guardG[\stateS]
	 \qqand[where]
  \]
  \begin{align*}
	 Y = & \{y \sst \exists i \qst y = y_i \text{ or $y$ is a parameter of an outgoing transition of } \stateS\}
	 \\[.5em]
	 \guardG[\stateS] = &
							  \begin{cases}
								 \mathsf{True} & \text{if } \stateS \text{ is accepting}
								 \\
								 \text{the disjunction of guards of the outgoing transitions of } \stateS & \text{otw}
							  \end{cases}
	 \\[.5em]
	 \guardG[\Assign] = & \bigwedge_{(\varCX := \exprE) \in \Assign}
								 \varCX = \exprE \ \land\ \bigwedge_{\varCX \not\in \Assign} \varCX =
								 \old\; \varCX
	 \\
							  & \qand[with ]\varCX \not\in \Assign \iff
								 (\varCY := \exprE) \in \Assign \implies \varX \neq \varY \qand \old\ \varCX \text{ does not occur in $\exprE$}
  \end{align*}
\end{frame}


\begin{frame}{Determinism}
  A \modelname is \emph{deterministic} if
  \[
	 \qqand[whenever]
	 \begin{tikzpicture}[dafsm, node distance = .5cm and 4cm]
		\node[state] (s) {};
		\node[state, above right = of s] (t) {};
		\node[state, below right = of s] (t') {};
		%
		\path (s) edge node {\dafsmlabel[{\guardG[1]}][{\modparty[1]}][@][@][{\Assign[1]}]} (t)
		  (s) edge node[below] {\dafsmlabel[{\guardG[2]}][{\modparty[2]}][@][@][{\Assign[2]}]} (t');
		\end{tikzpicture}
		\qqand[then]
		\guardG[1] \land \guardG[2] \implies \modparty[1] \partyConfl	 \modparty[2]
	 \]
	 \note{transitions from the same source state and calling the same function}
	 where $\_\partyConfl\_$ is the least binary symmetric relation s.t.
	 \[
		\new \partyConfl \modparty
		\qand \new \partyConfl \any[\partyP'][\roleR']
		\qand \roleR \neq \roleR' \implies \any \partyConfl \any[\partyP'][\roleR']
		\]
\end{frame}


\begin{frame}{Exercise: Determinism}
  The \modelname
  %
  $\dafsm = \raisebox{4pt}{
	 \tikz[dafsm, node distance= .25cm and 3cm]{
		\node[state] (S0)      {$\stateS[0]$};
		\node[state,draw=none] (dummy) [left =of S0] {};
		\node[state] (S1) [above right =of S0] {$\stateS[1]$};
		\node[state,accepting] (S2) [below right =of S0] {$\stateS[2]$};
		%
		\path
		(dummy) edge node[above] {\dafsmlabel[][\new][\init][\contractC][]} (S0)
		(S0) edge[bend right=10] node[below, near end] {$\alabel[1]$} (S1)
		(S1) edge[bend right=10] node[sloped, anchor=center, above]{$\alabel$} (S0)
		(S0) edge node[below] {$\alabel[2]$} (S2);
	 }
  }
  $
  \vfill
  is deterministic or not, depending on the labels $\alabel[1]$
  and $\alabel[2]$.
  \vfill
  \begin{enumerate}
  \item Is it the case that $\dafsm$ is not deterministic whenever
	 $\alabel[1] = \alabel[2]$?
  \item Find two labels $\alabel[1]$ and  $\alabel[2]$ that make $\dafsm$ deterministic
  \item Find two labels $\alabel[1] \neq \alabel[2]$ that make $\dafsm$ non-deterministic 
  \end{enumerate}
  \note{
	 \begin{enumerate}
	 \item no: eg for $\alabel[1] = \alabel[2] = \new$ $\dafsm$ is
		deterministic
	 \item $\alabel[1] = \alabel[2] = \dafsmlabel[][\new][@][@][]$ make
		$\dafsm$ deterministic because the next state is unambiguously
		determined by the caller which is fresh on both transitions
	 \item
		$\alabel[1] = \dafsmlabel[\varX \leq
		0][\partyP][@][\varX:\IntType][]$ and
		$\alabel[2] = \dafsmlabel[\varX \geq
		-1][\partyP][@][\varX:\IntType][]$ make $\dafsm$
		non-deterministic because the guards of $\alabel[1]$ and of
		$\alabel[2]$ are not disjoint therefore the next state is not
		determined by the caller
	 \end{enumerate}
  }
\end{frame}

\begin{frame}{Well-formedness}
  A \modelname is \emph{well-formed} when it is

  \itemit[@][3] empty-role free
  
  \itemit[@][5] consistent, and

  \itemit[@][7] deterministic
\end{frame}


\begin{frame}{Exercise: well-formedness}
  Which of the following \modelname is well-formed?

  \vfill
  
  \begin{tikzpicture}[dafsm, node distance = 3cm]
	 \node[state] (S0)      {$\stateS[0]$};
	 \node[] (S_) [left = 2.5cm of S0] {};
	 \node[state] (S1) [below right =of S0,yshift=2cm] {$\stateS[1]$};
	 \node[state] (S2) [above right =of S0,yshift=-2cm] {$\stateS[2]$};=
	 \node[state] (S3) [right=4.5cm of S0] {$\stateS[3]$};
	 \node[state,accepting] (S4) [right= of S3] {$\stateS[4]$};
	 %
	 \path
	 (S_) edge node {\dafsmlabel[][{\new[\partyB][\roleB]}][\init][\contractC][]}
	 (S0)
	 (S0) edge[bend right=10] node[below] {
		\dafsmlabel[][{\new[\partyO][\roleO]}][{\functionCF[1]}][][]
	 }
	 (S1)
	 (S0) edge[bend left=10] node[above] {
		\dafsmlabel[][{\new[\partyO][\roleR]}][{\functionCF[2]}][][]
	 }
	 (S2)
	 (S1) edge[bend right=10] node[below] {
		\dafsmlabel[][\partyO][{\functionCF[2]}][][]
	 }
	 (S3)
	 (S2) edge[bend left=10] node[above] {
		\dafsmlabel[][\partyO][{\functionCF[2]}][][]
	 }
	 (S3)
	 (S3) edge node {
		\dafsmlabel[][\partyO][{\functionCF[3]}][][]
	 }
	 (S4)
	 ;
  \end{tikzpicture}
  \note{
	 % 
	 yes: $\partyO$ is defined on paths it occurs on and the \modelname
	 is deterministic.
	 %
  }

  \vfill

  \begin{tikzpicture}[dafsm, node distance= 3cm]
	 \node[state] (S0)      {$\stateS[0]$};
	 \node[] (S_) [left of=S0, xshift = -2.5cm] {};
	 \node[state] (S1) [right=of S0] {$\stateS[1]$};
	 \node[state,accepting] (S2) [right of=S1 , xshift = 1cm] {$\stateS[2]$};
	 %
	 \path
	 (S_) edge node {
		\dafsmlabel[][{\new[\partyO][\roleO]}][\init][\contractC][\{\varCX := 1\}]
	 }
	 (S0)
	 (S0) edge node {
	  	\dafsmlabel[][\partyO][{\functionCF[1]}][][]
	 }
	 (S1)
	 (S1) edge node {
		\dafsmlabel[\varCX > 0][\partyO][{\functionCF[2]}][][]
	 }
	 (S2)
	 ;
  \end{tikzpicture}
  \note{\\[1em]
	 no: the transition from $\stateS[0]$
		violates~\cref{def:consistency} since $\mathsf{True}$ does not
		imply $\varCX > 0$ hinting that the protocol could get stuck in
		state $\stateS[1]$.
		%
		However, this % is not possible
    never happens because $\varCX$ is initially set to $1$
		and never changed, hence the transition from $\stateS[1]$ would be enabled
		when the protocol lands in $\stateS[1]$.
		%
	 }
\end{frame}
\hidden{

	\begin{example}\label{ex:semactic-correct}
	  The following \modelname is not well-formed
	  \[
		\]
	  %
		\finex
	\end{example}




%\hsl
%\subsection{Partipant \modelname}
%Participant checker checks if the caller is introduced in the current
% transition or in every path from the initial state to the 'from'
% state of the current transition. This verification is crucial to
% ensure that the FSM adheres to the rules governing participant
% introduction, thus maintaining its logical coherence. The process
% involves checking for the caller's direct introduction or through
% associated roles in the current transition and then analyzing all
% paths leading to the current transition's 'from' state to verify the
% caller's introduction at some point along each path.

%\begin{equation}
%	\text{t} = \left( C \in N_{T_c} \right) \lor \left( \bigwedge_{p \in P} \left( \exists t \in T_p : C \in N_t \right) \right)
%\end{equation}


 }
 
%%% Local Variables:
%%% mode: LaTeX
%%% TeX-master: "main"
%%% End:
