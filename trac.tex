\begin{frame}{Verification}
Checking well-formedness by hand is laborious and cumbersome

\vfill

So we implemented \thetool, which

\mytab transforms \modelnames in a DSL to specify \modelnames

\mytab verifies well-formedness condition relying on the SMT solver Z3
\end{frame}

\begin{frame}{The architecture of \thetool}
  %
  \note<1>{
	 %
	 the architecture of \thetool is compartmentalised into two principal modules:
	 \\
	 parsing and visualisation (yellow box) and
	 \\
	 \thetool's core (orange box).
	 %
	 The latter module implements well-formedness check (green box).
	 %
	 Solid arrows represent calls between components while dashed
	 arrows data IO.
	 %
  }
  \begin{tikzpicture}[node distance = .6cm and 1.1cm, remember picture]
	 % nodes
	 \node(txt) [data, minimum width=1.3cm] {\startingFile};
	 \node(ggen) [used, right = of txt] {\graphGen};
	 \node(vald) [dev, below = of txt, minimum width=1.3cm, yshift = -0.2cm] {\validator};
	 \node(call) [dev, below right = of ggen] {\callerChecker};
	 \node(nd) [dev, below = of call, yshift =.1cm, minimum width=2.2cm] {\detChecker};
	 \node(build) [dev, right = of nd] {\formulaBuilder};
	 \node(z3) [data, above = of build, yshift=.8cm] {\ztModel};
	 \node(zrun) [used, right = 1cm of z3] {\ztRunner};
	 % \node(toj) [dev, below = of json] {\toParser};
	 \node(gri) [dev, right = of vald, xshift = -0.2cm, yshift = -1cm] {\trGrinder};
	 \node(acons) [dev, below = of nd, minimum width=2.2cm] {\aConsistency};
	 \node(an) [dev, right = of acons, yshift = -.7cm] {\ztModelAnalyzer};
	 \node(verdict) [output, right = of an, xshift =-.5cm] {\analysisVerdict};
	 \node(vfms) [output, above = of zrun] {\fsmImage};
	 % edges
	 \path[dashed] (txt) -- (vald);
	 \path[dashed] (ggen.north) |- (vfms.west);
	 \path[dashed] (an) -- (verdict);
	 \node(dummy)[right = .3cm of vald]{};
	 \path[-, dashed] (vald) -- (dummy.center);
	 \path[dashed] (dummy.center) |- (ggen);
	 \path[dashed] (dummy.center) |- (gri);
	 %
	 \node(dummy2)[right = .3cm of gri]{};
	 %
	 \path[-] (gri) -- (dummy2.center);
	 \path (dummy2.center) |- (call);
	 \path (dummy2.center) -- (nd);
	 \path (dummy2.center) |- (acons);
	 %
	 \node(dummy21c)[above = .3cm of dummy2]{\mkNode{1}};
	 \node(dummy21d)[below = .35cm  of dummy21c]{\mkNode{2}};
	 \node(dummy21a)[below = .5cm of dummy2]{\mkNode{3}};
	 \node(dummy3)[right = .3cm of nd]{};
	 %
	 \path [dashed] (dummy3.center) -- (build);
	 \path[-, dashed] (call.east) -| (dummy3.center);
	 \path[-, dashed] (nd.east) -- (dummy3.center);
	 \path[-, dashed] (acons.east) -| (dummy3.center);
	 %
	 \path[dashed] (build) -- (z3);
	 \path[dashed] (z3) -- (zrun);
	 \node(dummy4)[right = .3cm of build,yshift = -.4cm]{};
	 \path[-] (zrun) |- (dummy4.center);, dashed
	 \path (dummy4.center) -| (an.north);
	 % Boxes
	 \begin{pgfonlayer}{background}
		\node[fit = (call) (nd) (acons),
		fill=green!70,
		opacity=.7,
		rounded corners,
		minimum width =2.8cm,
		minimum height = 3cm
		] {};
	 \end{pgfonlayer}
	 \begin{pgfonlayer}{background}
		\node[xshift =.35cm, fit = (dummy) (call) (acons) (build),
		fill=orange,
		opacity=.5,
		rounded corners,
		minimum width = 8.5cm,
		minimum height = 3.2cm,
		% label={[xshift=-3.5cm,yshift=-1cm, black] above:{\huge$\mathbf\circlearrowleft$}}
		] (ext){};
	 \end{pgfonlayer}
	 \draw [-stealth, very thin, double] (2.2,-1.3) arc(-160:140:.2);
	 \begin{pgfonlayer}{background}
		\node[fit = (vald) (call) (z3) (an.south east) (dummy4),
		fill=yellow!30,
		opacity=.5,
		rounded corners,
		minimum width = 11.5cm,
		minimum height = 4.3cm,
		% label={[xshift=-4cm,yshift=.5cm] below:{\thetool's core}}
		] {};
	 \end{pgfonlayer}
	 % Call outs
	 \node<2> [calloutnote, below = of vald, ball color=yellow, callout absolute pointer=(vald.south east)] {Syntactic checks};
	 %
	 \note<2>{basic syntactic checks on a DSL representation of
		\modelnames and transforming the input in a format that
		simplifies the analysis of the following phases:
		%
		\begin{itemize}
		\item passed to \graphGen for visual
		  representation of \modelnames (\fsmImage\ output)
		\item passed to the \trGrinder
		  component (orange box) for well-formedness checking.
	 \end{itemize}
  }
  %
  \node<3> [calloutnote, below = of gri, text width=3.5cm, ball color=orange!70, callout absolute pointer=(gri.south west)] {Well-formedness (iterative)};
  \node<4> [calloutnote, below = of acons, xshift=1cm, yshift=1cm,, ball color=orange!70, callout absolute pointer=(acons.west)] {Z3 model};
  \note<4>{
	 %
	 \aConsistency\ (arrow \mkNode{3}) to generate a Z3 formula which
	 holds if, and only if, the transtion is consistent.
	 %
  }
  \node<5> [calloutnote, below = of acons, xshift=1cm, yshift=1cm, ball color=orange!70, callout absolute pointer=(build.west)] {Z3 model};
  \note<5>{
	 %	 
	 computes the z3 f.la equivalent to the conjunction of the outputs which is then passed to a Z3 engine to check its satisfiability
  }
  \node<6> [calloutnote, below = .1cm of verdict, ball color=teal!50, callout absolute pointer=(an.east)] {analysis};
  \note<6>{
	 Finally, the \ztModelAnalyzer\ component
	 that diagnoses the output of Z3 and produces a \analysisVerdict\
	 which reports (if any) the violations of well-formedness of the
	 \modelname in input.
	 %
  }
  \end{tikzpicture}
\end{frame}

% \newcommand{\code}[1]{%\ \\\mytab \tt #1
%   \begin{block}{} #1 \end{block}}

\newenvironment<>{code}[2][]{%
  \setbeamercolor{block title}{fg=green,bg=black}%
  \setbeamerfont{block body}{family=\ttfamily}
  \setbeamercolor{block body}{fg=green,bg=black}
  \setbeamertemplate{block body code}[rose]
  \begin{block}{#1}#3{#2}
  }{\end{block}}


\begin{frame}[fragile]{Installation}
  Detailed instructions at \url{https://github.com/loctet/TRAC}

  \vfill
  
  Dependencies: Java RE (to render \modelname graphically) \& Python 3.6 or later
  %
%  \code{pip install z3-solver matplotlib numpy plotly pandas networkx}
  \note{forse servono solo solo z3-solver e networkx \\ https://doi.org/10.5281/zenodo.10996456}

  \begin{code}{}
	 pip install z3-solver matplotlib numpy plotly pandas networkx
  \end{code}
\end{frame}

%%% Local Variables:
%%% mode: LaTeX
%%% TeX-master: "main"
%%% End:
