\begin{frame}
  \frametitle{A nice sketch! \cite{azure,azure/smp}}
  
  A smart contract among Owners and Buyers
  \\[1em]
  \includegraphics[scale=.35]{/home/emilio/mypapers/bt4orc/coordination2024/papers/images/marketplace2.png}
  \\[1em]
  \textbf{initially} buyers can make offers
  \\\textbf{then}
  \\\phantom{xxx} \textbf{either} an owner can accept an offer and the protocol stops
  \\\phantom{xxx}	\textbf{or} the offer is rejected and the protocol restarts
\end{frame}

\begin{frame}
  \frametitle{What did we just see?}
  %
  A \underline{smart contract} looks like

  \itemit a \underline{choreographic model}
  \begin{quote}
	 global specifications determine the
	 enabled actions along the evolution of the protocol
  \end{quote}
  \itemit a \underline{typestate}
  \begin{quote}
	 In OOP, \quo{can reflects how the legal operations on imperative
		objects can change at runtime as their internal state
		changes.}~\cite{gtwa14}
  \end{quote}
\end{frame}

\begin{frame}
  \frametitle{A new coordination model}
  %
  So, we saw an interesting model where
  %
  \itemit
  distributed components coordinate through a global specification
  %
  \itemit which specifies which actions enabled along the computation
  %
  \itemit and it \quo{does not force} components to be cooperative!
  %
  \itemit \
\end{frame}

\begin{frame}
  \frametitle{Let's look again at our sketch}
    \begin{tikzpicture}
	 \useasboundingbox (3cm,2cm)rectangle(-3.3cm,-1.8cm);
	 \node (smp) {
		\includegraphics[scale=.35]{/home/emilio/mypapers/bt4orc/coordination2024/papers/images/marketplace2.png}
	 };
	 \node<2> (txt) [right = .5cm of smp, fill=orange!20, opacity=.5, yshift=.2cm, text width = 7.3cm, scale = .9, align=left]{%
		\begin{minipage}[t][5cm][t]{\textwidth}
		  but...%
		  \cons can buyers be owners too?%
		  \note{ok\\[2em]}%
		  \cons what's the difference between \underline{roles} and
		  \underline{instances}?%
		  \note{ok\\[2em]}%
		  \cons what's the scope and and quantification?%
		  \note{from~\cite{azure/smp}: “The transitions between the
			 \textcolor{red}{Item Available} and the
			 \textcolor{red}{Offer Placed} states can continue until the
			 owner is satisfied with the offer made.”  so, after a
			 rejection, the new offer must be from the original buyer or
			 a new one?\\[2em]}%
		  \cons when are transitions enabled?%
		  \note{ok\\[2em]}%
		  \cons how does the state of the contract change?%
		  \note{should the price of the item remain unchanged when the
			 owner invokes the Reject?}%
		\end{minipage}
	 };
  \end{tikzpicture}
\end{frame}

\begin{frame}
  \frametitle{...and by the way}

  \begin{columns}[t]
	 \begin{column}{.45\linewidth}
		\includegraphics[scale=.2]{solidity_and_FM}

		\url{https://medium.com/@solidity101/formal-verification-of-smart-contracts-in-solidity-192f2a4d0abd}
	 \end{column}
	 \begin{column}{.45\linewidth}
		\includegraphics[scale=.2]{ethereum_FM}

		\url{https://ethereum.org/en/developers/docs/smart-contracts/formal-verification/}
	 \end{column}
  \end{columns}
\end{frame}

\begin{frame}{Let's go formal!}
  Our first attempt was to reuse \quo{our toolboxes}, but

  \cons roles with multiple instances

  \cons instances with many roles

  \cons do the known notion of well-formedness make sense?

  \cons data-awareness is crucial

  \pause\vfill
  So we had to came up with some new behavioural types.
\end{frame}

%%% Local Variables:
%%% mode: LaTeX
%%% TeX-master: "main"
%%% End:
